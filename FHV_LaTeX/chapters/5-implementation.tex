\chapter{Implementation}
\section{System Architecture and Technical Stack}
\section{Implementation of Selected Approach}
\section{Data Collection and Preprocessing}


1. Data Preparation: Ensure your dataset includes diverse outfits with corresponding aesthetic scores.
2. Normalization: Normalize input images to ensure consistency.
3. Hyperparameter Tuning: Experiment with different learning rates, batch sizes, and number of epochs.
4. Validation Strategy: Use a validation set to monitor performance and prevent overfitting.

Data Collection: The first step is to gather labeled data, which typically consists of input features and their corresponding target labels. This data should be representative of the problem you want to solve.
Data curation: The process of cleaning and organizing the collected data to ensure its quality and reliability. This step involves removing any outliers or inconsistencies, handling missing values, and transforming the data into a suitable format for training the model.
Data Splitting: The collected data is usually divided into two subsets: the training dataset and the test data. Train the model with the training dataset, while the test data is reserved for evaluating its performance.

\section{AI Model Development and Training}

Model Selection: Depending on the problem at hand, you choose an appropriate supervised learning algorithm. For example, if you're working on a classification task, you might opt for algorithms like logistic regression, support vector machines, or decision trees.
Training the Model: This step involves feeding the training data into the chosen algorithm, allowing the model to learn the patterns and relationships in the data. The training iteratively adjusts its parameters to minimize prediction errors with its learning techniques.
Model Evaluation: After training, you evaluate the model's performance using the test set. Standard evaluation metrics include accuracy, precision, recall, and F1-score.
Fine-tuning: If the model's performance is unsatisfactory, you may need to fine-tune its hyperparameters or consider more advanced algorithms. This step is crucial for improving the model's accuracy.
Deployment: Once you're satisfied with the model's performance, you can deploy it to make predictions on new, unseen data in real-world applications.

Fine-Tuning Techniques

\href{https://viso.ai/deep-learning/pose-estimation-ultimate-overview/}{Human Pose Estimation reference}

SAM
category-guided attention mechanisms


\href{https://encord.com/blog/mastering-supervised-learning-a-comprehensive-guide/}{Supervised learning reference}
\href{https://blog.dataiku.com/outfit-recommendation-system}{Building an AI-Powered Outfit Recommendation System With Dataiku}
\href{https://nisargdoshi.medium.com/smart-fashion-recommendation-using-resnet50-b21d47cc91b1}{Smart Fashion Recommendation using ResNet50}