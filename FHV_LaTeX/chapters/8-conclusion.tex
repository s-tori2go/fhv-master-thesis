\chapter{Conclusion and Future Work}
\section{Summary of Findings}
\section{Future Research Directions}


explore the possibility of implementing different types of feedback (Explicit + Implicit -> Rating Matrix). \href{https://learnopencv.com/recommendation-system/}{Mastering Recommendation System: A Complete Guide}

Additionally, insights gained during the experimentation phase will be documented, particularly regarding the trade-offs between simplicity (single-input systems) and complexity (multi-input systems). These findings are anticipated to inform future research directions in \acs{AI} and fashion, particularly in areas such as recommendation systems, virtual styling assistants, and augmented reality try-ons.

Future Research Directions (For Thesis): Investigate the use of machine learning to *dynamically* adjust compatibility rules based on the current outfit's characteristics. Develop a system that allows users to provide more detailed style preferences and integrate these preferences into the generation process. Extend the system to generate outfits that synthesize elements from multiple existing style guidelines, creating unique and novel combinations. - While the output is visually appealing, integration notes suggest a further refinement of the compatibility criteria, potentially incorporating factors like color palettes, silhouette considerations, and formality level.  The system performs well with common garment types - shirts, pants, dresses - but struggles with more complex combinations.
