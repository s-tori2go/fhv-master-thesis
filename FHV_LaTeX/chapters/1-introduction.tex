\chapter{Introduction}

The integration of \ac{AI} into the fashion industry has opened new doors for innovation and has transformed key areas such as design, production, retail and marketing. Within this rapidly evolving landscape, one particularly interesting application is personalized styling. Specifically, the use of \acs{AI} to evaluate (and recommend) fashion outfits tailored to individual preferences. This thesis investigates the application of \acs{DL} techniques to assess the visual quality of fashion ensembles, with the central research question being:

\begin{displayquote}
\textbf{How can existing \acs{DL} models be used to give a score representing evaluation of visual compatibility of a fashion outfit based on images of individuals wearing clothes?}
\end{displayquote}

This introductory chapter describes the initial situation. It outlines the motivation behind the research question, the objectives and the problem statement. The chapter concludes with a summary of the scope and expected outcomes.

\section{Motivation}

The motivation for this thesis lies in the potential of \acs{AI} to address gaps in personalized fashion. Current \acs{AI}-driven fashion applications often focus on generic outfit recommendations, neglecting the nuanced interplay between visual aesthetics, personal preferences and contextual factors. There exists an opportunity to develop systems that, on the one hand, understand general styling guidelines and recognize patterns in fashion styles and, on the other hand, adapt to individual tastes and situational demands. By developing a robust \acs{AI}-powered outfit evaluation system, the groundwork is laid for creating a personalized \acs{AI}-powered stylist application. Such an application could not only assess outfits but also provide tailored recommendations that align with individual preferences and contextual requirements. This thesis represents a foundational step toward understanding how \acs{AI} can be leveraged to enhance user experiences in the fashion domain, satisfying the growing demand for personalized fashion solutions in our digital age.

\section{Objectives and Problem Statement}

This thesis addresses four significant gaps in the field:

\begin{enumerate}
\item \acs{AI} models that assess outfits while factoring in individual features are still relatively unexplored. \cite[cf.]{chen_survey_2023}, \cite[cf.]{deldjoo_review_2022}
\item Real-world outfits typically consist of multiple items such as shoes, accessories and more. However, existing methods assume a fixed input size and evaluate the compatibility between only two items (e.g. top and bottom) for a specific user. As a result, they are less capable of evaluating outfits with multiple or inconsistent item counts. \cite[cf.]{chen_survey_2023}
\item Most models rely on predefined category labels to learn and evaluate outfit compatibility. But in real life, fashion items often belong to overlapping or unclear categories (e.g. hybrid pieces such as shirt-jackets or casual-formal dresses). There is a need for models that can learn compatibility based on visual, contextual or semantic features without depending on fixed category labels. \cite[cf.]{chen_survey_2023}
\item Existing approaches often fail to account for the interplay between visual aesthetics, contextual factors and personal preferences. \cite[cf.]{chen_survey_2023}, \cite[cf.]{deldjoo_review_2022}
\end{enumerate}

This work tackles these challenges by introducing an approach to evaluate how good an outfit looks on a person, using photos of them wearing it. The proposed approach leverages existing \acs{DL} techniques to assess the compatibility of an outfit. It can handle any number of clothing items and does not rely on fixed labels, allowing for optional integration of supplementary data.

To achieve this, the thesis on hand is structured around a few core steps. The first involves analyzing existing research and the \acs{DL} models it employs. This is done to assess their relevance for outfit evaluation as well as to identify models of use to the presented use case. Consequently, the second step involves developing a concept of a solution while tackling the limitations and research gaps identified earlier. In the third step suitable models are selected and integrated into a pipeline. This pipeline processes input images and generates a numerical score that reflects the visual quality of the outfit. Afterwards, experiments are conducted to evaluate the effectiveness of the implementation alone as well as with incorporated data on contextual factors and personal preferences.

\section{Scope and Expected Outcomes}

The scope of this thesis encompasses several key areas within \acs{AI} and fashion technology. Technically, the study focuses on the utilization and adaptation of existing state-of-the-art \acs{DL} models for image-based outfit evaluation. It includes experimenting with different types of \acs{ML} models within \acs{CV}, transfer learning, feature extraction methods, embedding techniques and evaluation metrics to assess visual elements. Functionally, the project involves developing a pipeline that accepts images as input and outputs a numerical score reflecting the visual quality of the outfit. Scientifically, the research analyzes the effectiveness of combining different model architectures in capturing subjective aesthetic judgments.

The expected outcomes of this master's thesis include the development of a proof-of-concept prototype that demonstrates the feasibility of \acs{AI}-driven outfit evaluation. The computational solution is capable of analyzing images of individuals wearing outfits and assigning a score that reflects the overall aesthetic appeal and visual coherence of the look. This prototype will serve as the foundation for a personalized \acs{AI}-stylist application, enabling users to receive real-time feedback on their outfits and access tailored recommendations.